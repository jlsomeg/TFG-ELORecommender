\chapter{Introducción}
\label{cap:introduccion}

\chapterquote{Nunca dejes que tu sentido de la moralidad te impida hacer lo que está bien}{Los límites de la Fundación, Isaac Asimov (1982)}

\section{Motivación}
Los sistemas de recomendación son elementos invisibles, discretos y opacos, que la mayoría de personas no conocen (si quiera si existen) pero todo el mundo ha usado alguna vez en su vida. Con el auge de internet y la capacidad de procesamiento, la ascensión de cantidad de datos a los que un usuario tiene acceso son, inequívocamente hablando, infinitos. Términos de los que hace años se habla como sobreinformacion (meter referencia https://www.reuters.com/article/net-us-technology-mobile-poll/online-sharing-information-overload-is-worldwide-problem-poll-idUSBRE8840NF20120905), o con aplicaciones en tiempos mas recientes como Big Data, denotan la cantidad de datos con los que los usuarios tienen que lidiar a la hora de realizar elecciones. \\

Tanto el problema de la saturación, como la solución de la recomendación, no son términos que se hayan inventado en este siglo. Que libros de novela gótica debería empezar a leer si quiero adentrarme en este genero literario, eran relativamente resueltos si empiezas con los mas famosos, o mejor aun, los que te pueda recomendar algún conocido versado en el genero y que conoce de alguna manera tus gustos y disgustos. Los sistemas de recomendación automatizados se diferencian de este conocido en una cosa. La gran capacidad de procesamiento y obtención de datos.  Mediante esto, los recomendadores son capaces de recomendarte elementos y guiarte por caminos que ni sabias que existían, pero que tus interacciones denotan relación con ellos. \\

Existen multitud de sistemas de recomendación que se aplican en variedad de escenarios y entornos, la mayoría de estos de intereses relacionados con el ocio o el comercio electrónico (e-commerce) dado que sus origenes empiezan ahi (meter referencia Brent Smith,Greg Linden,\textit{Two Decades of Recommender Systems at Amazon.com},2017). 

Pero en lo referente al uso de estos sistemas aplicados a sistemas educacionales adaptados, poco se ha desarrollado en un nicho aun por explotar. Estos escenarios presentan un problema parecido a los vistos anteriormente. La traducción está, en que la cantidad de objetos de aprendizaje se asemeja a la sobrecarga de información que hablábamos anteriormente. \\

Las nuevas tecnologías y el rápido crecimiento de Internet ha facilitado el acceso a la información para todo tipo de personas, planteando nuevos retos a la educación al utilizar Internet como medio. Uno de los mejores ejemplos es cómo guiar a esos estudiantes en sus procesos de aprendizaje. La necesidad de buscar la orientación de sus profesores u otros compañeros que muchos usuarios de Internet experimentan cuando esforzarse por elegir sus lecturas, ejercicios o prácticas es un realidad muy común. Con el fin de atender a esta necesidad se han desarrollado muchas estrategias de información y recomendación. Los sistemas de recomendación intentan ayudar al usuario, presentándolo. aquellos objetos en los que podría estar más interesado, en base a sus conocidos preferencias o en las de otros usuarios con similares características.(meter referencia Oscar Sanjuán Martínez,Cristina Pelayo G-Bustelo,Rubén González Crespo,Enrique Torres Franco 2,\textit{Using Recommendation System for E-learning Environments at degree level},2009). \\


En este trabajo, realizaremos una investigación que se centrará en diseñar un sistema de recomendación de problemas para los usuarios de la plataforma online ¡Acepta el Reto! que se aproxime lo máximo al nivel de habilidad, exigencia e intereses del mismo y de como implementar su funcionalidad en una meta-aplicacion web donde se pueda observar la información mas reseñable.


\section{Antecedentes}

Nuestros antecedentes para realizar esta investigación vienen precedidos por la evolución de los sistemas de recomendación comentado anteriormente, y lo útiles que podrían ser en su aplicación a entornos educativos (en este caso e-lerning) como concluye el articulo (meter referencia Oscar Sanjuán Martínez,Cristina Pelayo G-Bustelo,Rubén González Crespo,Enrique Torres Franco 2,\textit{Using Recommendation System for E-learning Environments at degree level},2009). \textit{"Nuestra investigación muestra que el problema de la sobrecarga de información también está presente en entornos educativos a distancia. Los resultados obtenidos muestran que la mayoría de los usuarios no están dispuestos o no puede hacer todas las prácticas que el sistema pone a su disposición, es por eso que encontrarían la ayuda útil para decidir qué prácticas deberían hacer} \\

¡Acepta el reto! se presenta como un juez online en el que los autores ponen a disposición de los usuarios problemas en español planteados específicamente para el aprendizaje por parte de los alumnos de las diferentes asignaturas de los Grados de Informática y Ciclos Formativos. Dispone de una batería de unos 300 problemas, que pueden recorrerse por categorías, lo que simplifica encontrar aquellos que resulten más adecuados en cada momento. Muchos de los problemas están pensados para forzar soluciones con complejidades específicas no solo en tiempo, sino también en espacio, algo poco habitual en otros jueces.(meter referencia ¡Acepta el reto!: juez online para docencia en español Pedro Pablo Gómez-Martín, Marco Antonio Gómez-Martín)

A todos ojos, es un software que encaja perfectamente con el modelo de e-learning y que ademas, presenta la dificultad de la sobreinformacion al poseer en sus repositorios miles de ejercicios de programación, lo que genera el problema al usuario (sobretodo a los primerizos) de la elección de cual seria el problema que mas se ajusta a su nivel de conocimiento. En ¡Acepta el reto!, el único sistema de guía existente es la categorizacion de sus problemas en distintos ámbitos técnicos de la programación, elemento insuficiente para solucionar el problema. 

A este problema se le intento poner solucion en el trabajo realizado en 2016 por profesores de la Facultad de Informática (UCM) y miembros del grupo de investigación GAIA en el articulo de Jimenez-Diaz et al. "Similarity Metrics from Social Network Analysis for Content Recommender Systems", del congreso International Conference on Case-Based Reasoning (ICCBR) de 2016. En el, se propone un método de recomendación para jueces en línea a partir de un grafo de interacción de problemas contenidos en el juez para guiar estos usuarios. A este trabajo le sucede el TFM de Marta Caro Martinez "Sistemas de Recomendación basado en técnicas de predicción de enlaces para jueces en línea" en el que reproducía y ampliaba la investigación  de Guillermo Jimenez Diaz incluyendo nuevas propuestas de recomendadores basadas en el grafo de interacción de los usuarios de la plataforma, así como la inclusión de nuevos métodos que cambian la forma de recomendar problemas a los usuarios.

Nuestro trabajo se basa en lograr el mismo objetivo de los trabajos anteriores en lo referido a establecer un sistema de recomendación de problemas en ACR, pero tomando caminos distintos a los de Guillermo y Marta, planteando una solucion desde otros modelos de recomendación para explorar el abanico de posibilidades de atacar este problema.


\section{Objetivos}

El objetivo que se persigue en este trabajo sera: \\
	
	1. Investigación sobre sistemas de recomendación, jueces online, algoritmos de emparejamiento y métodos estadísticos \\
	
	2. Planteamiento, diseño y desarrollo de un método de emparejamiento y recomendación \\
	
	3. Adaptación y diseño del método anterior a las necesidades,condiciones y  utilidades de ¡Acepta el reto! \\
	
	4. Desarrollo, experimentación y análisis de los resultados obtenidos,usando un conjunto de datos extraídos de ACR, en orden de ajustar el método a la mayor respuesta de calidad posible \\
	
	5. Diseño e integración de una aplicación web donde poner en funcionamiento el sistema de recomendación y muestreo de información de problemas y usuarios \\
	
\section{Organización de la memoria}	